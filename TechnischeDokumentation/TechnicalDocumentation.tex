\documentclass[10pt,a4paper]{article}
\usepackage[utf8]{inputenc}
\usepackage{amsmath}
\usepackage{amsfonts}
\usepackage{amssymb}
\usepackage{german}
\usepackage{fancyhdr}
\usepackage{graphicx}
\usepackage{geometry}
\usepackage{listings}
\usepackage{hyperref}
\usepackage[onehalfspacing]{setspace}
\usepackage{color}
\usepackage[usenames,dvipsnames]{xcolor}
\usepackage{DejaVuSans}
\usepackage[T1]{fontenc}

\renewcommand*{\familydefault}{\sfdefault}
\geometry{verbose,a4paper,tmargin=35mm,bmargin=35mm,lmargin=25mm,rmargin=25mm}
\author{Dominik Heeb, Fabian Keller}
\title{Dynamic Paralle Checker}
\pagestyle{fancy}
\fancyhead{}
\fancyhead[L]{Dynamic Paralle Checker}
\fancyhead[R]{Domink Heeb, Fabian Keller}
\fancyfoot{}
\fancyfoot[R]{Seite \thepage}

\definecolor{bluekeywords}{rgb}{0,0,1}
\definecolor{greencomments}{rgb}{0,0.5,0}
\definecolor{redstrings}{rgb}{0.64,0.08,0.08}
\definecolor{xmlcomments}{rgb}{0.5,0.5,0.5}
\definecolor{types}{rgb}{0.17,0.57,0.68}
\lstset{language=[Sharp]C,
captionpos=b,
showspaces=false,
showtabs=false,
breaklines=true,
showstringspaces=false,
breakatwhitespace=true,
escapeinside={(*@}{@*)},
commentstyle=\color{greencomments},
morekeywords={partial, var, value, get, set},
keywordstyle=\color{bluekeywords},
stringstyle=\color{redstrings},
basicstyle=\ttfamily\normalsize,}

\usepackage{amsmath}
\usepackage[some]{background}
\usepackage{lipsum}

\definecolor{titlepagecolor}{cmyk}{1,.20,0,.25}

\DeclareFixedFont{\bigsf}{T1}{phv}{b}{n}{1.5cm}

\backgroundsetup{
scale=1,
angle=0,
opacity=1,
contents={\begin{tikzpicture}[remember picture,overlay]
 \path [fill=titlepagecolor] (-0.5\paperwidth,5) rectangle (0.5\paperwidth,10);  
\end{tikzpicture}}
}
\makeatletter                       
\def\printauthor{%                  
    {\LARGE Studenten:\\\vspace{10pt}
    \large \@author \\\vspace{20pt}
    \LARGE Dozent:\\\vspace{10pt}
    \large Prof. Dr. Luc Bläser \\
	\texttt{lblaeser@hsr.ch}}}            
\makeatother
\author{%
    Fabian Keller \\
    Semester 5 \\
    \texttt{f3keller@hsr.ch}\vspace{15pt} \\
    Dominik Heeb \\
    Semester 5 \\
    \texttt{d1heeb@hsr.ch}
    }
\renewcommand{\listfigurename}{}
\begin{document}
\begin{titlepage}
\BgThispage
\newgeometry{left=1cm,right=4cm}
\vspace*{2cm}
\noindent
\textcolor{white}{\bigsf Dynamic Parallel Checker\\[0.5cm] \begin{huge}Semesterarbeit - Technische Dokumentation\end{huge}}
\vspace*{2.0cm}\par
\noindent
\begin{minipage}{0.35\linewidth}
    \begin{flushright}
        \printauthor
    \end{flushright}
\end{minipage} \hspace{15pt}
%
\begin{minipage}{0.02\linewidth}
    \rule{1pt}{300pt}
\end{minipage} \hspace{40pt}
%
\begin{minipage}{0.6\linewidth}
\begin{center}
\begin{huge}
Eine Studie über Dynamic Parallel Checking Methoden
\end{huge}
\end{center}
\end{minipage}
\end{titlepage}
\restoregeometry

\newpage
\tableofcontents 
\newpage

\section{Abstract}
\begin{flushleft}
In diesem Projekt werden verschiedene Methoden zur dynamischen Analyse von Parallelem Code evaluiert. Dafür wird ein Dynamic Parallel Checker Prototyp entwickelt, der während der Laufzeit überprüft ob Nebenläufigkeitsfehler (Race Conditions) auftreten. Ebenfalls Inhalt dieses Projekts ist es, einen eigenen Algorithmus zu entwickeln, der die Laufzeitanalyse ermöglicht. Diesbezüglich haben wir uns für einen Algorithmus entschieden, der mit Hilfe einer Vector Clock eine partielle Ordnung innerhalb der Lese- und Schreibzugriffe von den verschiedenen Threads herstellen kann.\\

\end{flushleft}
\section{Technischer Bericht}
\subsection{Einleitung und Übersicht}
\subsubsection{Dynamic Checker}
\begin{flushleft}
Um Race Conditions (sh. \ref{race_conditons}) zu erkennen, gibt es verschiedene Ansätze. Zwei davon sind, der Dynamic Checker und der Static Checker. In diesem Projekt
wird ein Dynamic Checker entwickelt. Ein dynamic Checker hat die Aufgabe, Race Conditions zur Laufzeit (dynamic) zu erkennen. Der Static Checker
im Gegensatz behandelt das Erkennen zur Entwicklungszeit. Um einen Dynamic Checker realisieren zu können, muss ein fertiges Programm instrumentiert werden.
Mit Instrumentation ist gemeint, dass der Kompilierte Code angepasst wird, so dass er während der Laufzeit einen Erkennungsalgorithmus ausführen kann, jedoch nicht das Verhalten des Programms verändert.\\
Der Checker analysiert hauptsächlich Lese- und Schreibzugriffe auf Variablen. Um die Präzision zu erhöhen, müssen auch Lock/ Unlock und Thread.Start usw. ausgelesen werden.\\
Mehr informationen dazu unter: \ref{vector_algorithm} Vector Clock Algorithmus
\end{flushleft}
\subsubsection{Vector Clock}

\subsubsection{Happened-Before Beziehung}
\begin{flushleft}
Um mit Hilfe der Vector Clock die Nebenläufigkeit von Lese- und Schreibzugriffe zu bestimmen, verwenden wir die Happened-Before Beziehung von Leslie B. Lamport. Diese Beziehung wird in der Lamport Clock und in der Vector Clock verwendet um eine partielle Ordnung innerhalb mehreren Ereignissen herzustellen. Bei neben läufigen Programmen kann keine totale Ordnung erreicht werden, darum muss mit einer partiellen Ordnung gearbeitet werden.\\
Diese Partielle Ordnung dient der Überprüfung, ob gewisse Lese- und Schreibzugriffe neben läufig oder sequenziell abgelaufen sind. Dies ist wichtig für unseren Algorithmus, denn der kümmert sich nur um die neben läufigen Ereignissen.\\
Jeder Thread oder Task besitzt eine aktuelle Vector Clock, die jedem Lese- und Schreibzugriff zugewiesen wird. Bei einem Synchronisationspunkt wird diese Vector Clock synchronisiert und die neuen Zugriffe erhalten eine neue Vector Clock. Die Happened-Before Beziehung ermöglicht es dann eine Aussage zu den Beziehungen zwischen den einzelnen Lese- und Schreibzugriffe der unterschiedlichen Thread oder Tasks zu machen.\\[0.5cm]
Eigenschaften der Happened-Before Beziehung:
\begin{itemize}
\item Auf demselben Thread oder Task: a -> b wenn die Zeit von a < b. (Zeit ist durch Vector Clock gegeben)
\item Wenn eine Synchronisation zwischen zwei Threads oder Tasks durchgeführt wurde, dann a -> b wenn a der Thread oder Task ist von dem aus synchronisiert wird und b der Thread oder Task ist zu dem synchronisiert wird.
\item Für drei Zugriffe mit Synchronisation a, b, c, wenn a -> b und b -> c, dann a -> c (Transitivität)
\end{itemize}
Unser Algorithmus vergleicht jede Komponente der Vector Clock von Zugriff a mit der passenden Komponente der Vector Clock von Zugriff b. Dabei können zwei verschiedene Beziehungen bestehen:\\[0.5cm]
\textbf{{\large Happened Before}}\\
Test
\\[0.5cm]
\textbf{{\large Concurrent}}\\
Test
\end{flushleft}
\subsubsection{Race Condition}\label{race_conditons}
\begin{flushleft}
Bei Race Conditions handelt es sich um Speicherzugriffsfehler. Es gibt 2 Stufen von Race Conditions: Data Races und Semantisch höhere Race Conditions.\\
Beide Kategorien entstehen durch Synchronisationsfehlern zwischen zwei Threads, welche auf die gleiche Ressource zugreiffen.
\begin{figure}[h]
\centering
\begin{tabular}{|cc|}
\hline
\multicolumn{2}{|c|}{Konto = 100} \\ 
 &  \\ 
Thread 1 & Thread 2 \\ 
Konto + 200 & Konto - 100 \\ 
\hline
\end{tabular}
\caption[Beispiel Race Condition]{Ein unsynchronisierter Zugriff auf ein Konto}
\label{fig:exampleRaceCondition}
\end{figure}\\
Unter \autoref{fig:exampleRaceCondition} ist ein typisches Beispiel einer Race Condition beschrieben. Der Befehl + oder - setzt sich folgendermassen zusammen:
Lesen, Addieren, Speichern.\\
Wenn zwei Threads dies parallel Ausführen, ist nicht deterministisch welcher Thread wann durchgeführt wird. Daher kann die Situation entstehen bei der zuerst Thread 1 liest, dann Thread 2 liest und erst danach Thread 1 addiert und speichert. Thread 2 hat dann immernoch den alten Wert in seiner Berechnung und überschreibt das Konto dann anstatt mit 200 mit 0. Dies ist eine typische Race Condition
\end{flushleft}

\subsection{Vektor Clock Algorithmus}\label{vector_algorithm}
\subsubsection{Vector Clock pro Thread}
\subsubsection{Lock-History}
\subsubsection{Funktion}
\subsection{Implementation}
\subsection{Schlussfolgerungen}
\subsection{Backlog}
\section{Glossar}
- Partielle Ordnung\\
- Totale Ordnung
\section{Abbildungsverzeichnis}
\listoffigures
\section{Literaturverzeichnis}
\renewcommand{\section}[2]{}%
\begin{thebibliography}{xxxxxxxxxxxxx}
\bibitem[BMBF, 2003]{bmbf}"'IT-Ausstattung der allgemein bildenden und berufsbildenden 
                         Schulen in Deutschland"', http://www.schulen-ans-netz.de/   
                         neuemedien/fakten/dokus/it-ausstattung-2003.pdf, 10.03.2005	
\end{thebibliography}
\end{document}